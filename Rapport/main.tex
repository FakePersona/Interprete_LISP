\documentclass[a4paper,11pt]{article}
\usepackage{a4wide}%

\usepackage{tabularx}

\usepackage{fullpage}%
\usepackage[T1]{fontenc}%
\usepackage[utf8]{inputenc}%
\usepackage[main=francais,english]{babel}%

\usepackage{graphicx}%
\usepackage{xspace}%
\usepackage{float}

\usepackage{url} \urlstyle{sf}%
\DeclareUrlCommand\email{\urlstyle{sf}}%

\usepackage{mathpazo}%
\let\bfseriesaux=\bfseries%
\renewcommand{\bfseries}{\sffamily\bfseriesaux}

\newenvironment{keywords}%
{\description\item[Mots-clés.]}%
{\enddescription}


\newenvironment{remarque}%
{\description\item[Remarque.]\sl}%
{\enddescription}

\font\manual=manfnt
\newcommand{\dbend}{{\manual\char127}}

\newenvironment{attention}%
{\description\item[\dbend]\sl}%
{\enddescription}

\usepackage{listings}%

\lstset{%
  basicstyle=\sffamily,%
  columns=fullflexible,%
  language=c,%
  frame=lb,%
  frameround=fftf,%
}%

\lstMakeShortInline{|}

\parskip=0.3
\baselineskip

\sloppy

%opening
\title{Interprète Lisp en C++}
\author{Antonin Garret \and Rémy Sun}
\date{21 avril 2016}


\begin{document}

\maketitle

\section*{Interprète dynamique}

Dans un premier temps, nous avons implémenté, à partir du code fourni, un interprète qui possède les principale fonctions Lisp, avec une gestion dynamique de l'environnement

\subsection*{Toplevel}

Nous avons encapsulé la gestion des entrées dans un \texttt{toplevel}, appelé par la fonction \texttt{main}. Ce toplevel gère la directive setq, qui permet de modifier l'environnement. Elle fait appel à une fonction \texttt{add_new_binding} que nous avons implémenté, qui permet de créer un nouvel objet \texttt{Binding} (qui comporte deux éléments : un symbole et la valeur qui lui est associée) que l'on ajoute à l'environnement courant.



\begin{itemize}
\item http://stackoverflow.com
\item https://openclassrooms.com
\item http://www.cs.rpi.edu
\end{itemize}


\end{document}
