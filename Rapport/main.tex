\documentclass[a4paper,11pt]{article}
\usepackage{a4wide}%

\usepackage{tabularx}

\usepackage{fullpage}%
\usepackage[T1]{fontenc}%
\usepackage[utf8]{inputenc}%
\usepackage[main=francais,english]{babel}%

\usepackage{graphicx}%
\usepackage{xspace}%
\usepackage{float}

\usepackage{url} \urlstyle{sf}%
\DeclareUrlCommand\email{\urlstyle{sf}}%

\usepackage{mathpazo}%
\let\bfseriesaux=\bfseries%
\renewcommand{\bfseries}{\sffamily\bfseriesaux}

\newenvironment{keywords}%
{\description\item[Mots-clés.]}%
{\enddescription}


\newenvironment{remarque}%
{\description\item[Remarque.]\sl}%
{\enddescription}

\font\manual=manfnt
\newcommand{\dbend}{{\manual\char127}}

\newenvironment{attention}%
{\description\item[\dbend]\sl}%
{\enddescription}

\usepackage{listings}%

\lstset{%
  basicstyle=\sffamily,%
  columns=fullflexible,%
  language=c,%
  frame=lb,%
  frameround=fftf,%
}%

\lstMakeShortInline{|}

\parskip=0.2
\baselineskip

\sloppy

%opening
\title{Interprète Lisp en C++}
\author{Antonin Garret \and Rémy Sun}
\date{24 avril 2016}


\begin{document}

\maketitle

\begin{abstract}
  L'interprétation du langage Lisp par un langage non fonctionnel comme C++
  demande d'adapter la gestion intuitive des structures mises en jeu. En
  implémentant une liaison lexicale et une clôture à la manière de
  l'implémentation Scheme de Lisp, nous avons particulièrement travaillé sur la
  gestion de l'environnement. Par
  des tests d'implémentation de fonctions, nous avons ensuite démontré la bonne
  marche de notre interprète.
\end{abstract}

\section*{Introduction}

Le Lisp est un langage fonctionnel interprété permettant des implémentations
différentes. De telles implémentations, codées en Ocaml ont été vues en TD de
programmation. Ocaml étant un langage fonctionnel, cette implémentation est
relativement naturelle.

Au travers de ce projet, nous nous intéressons à l'implémentation d'un
interprète Lisp en C++, ce qui pose de sérieuses questions puisque le langage
d'implémentation n'est alors plus un langage qui peut traiter les fonctions
comme des citoyens de première classe. Nous nous sommes plus particulièrement attachés à
l'implémentation d'une gestion d'environnements et de clôtures proches de celle
très souple de Scheme. 

Deux versions de code sont proposées dans le code rendu. La version définitive
est le code \texttt{Scheme}, mais le code \texttt{Copie} relève d'une version du
code où nous avions voulu explorer une implémentation de la clôture différente de 
celle
qui avait été étudiée en TD. Cela a été l'occasion de réaliser les
grandes limitations de cette implémentation, et pourquoi l'implémentation
proposée en TD était grandement préférable et généralement celle choisie.

Une documentation du code \texttt{Scheme} générée par Doxygen est proposée avec
le code.

\section{Objectif}

L'objectif que nous nous sommes fixés pour ce projet était de construire un
interprète Lisp gérant les liaisons et clôtures à la manière de \texttt{Scheme}. Plus
particulièrement, nous étions intéressés par la gestion de liste chaînées de
frames, et par la façon dont on pouvait représenter ces frames pour permettre la
création d'un tel interprète.

Au delà des problèmes de définitions circulaires posés, le grand défi a été de
gérer correctement la création des copies (implicites dès qu'on passe en
argument de fonction) et la bonne détection des clôtures.

Nous nous somme fixés les objectifs intermédiaires suivants:
\begin{itemize}
\item Compléter le squelette fourni pour obtenir un interprète dynamique, ce qui
  constitue une première étape facilement vérifiable.
\item Passer de la gestion d'environnement proposée à une gestion par listes
  chaînées, ce qui permet au passage de commencer à réfléchir à la façon de
  gérer les frames.
\item Créer un premier interprète à liaison statique identique à celui vu en TD.
\item Implémenter une gestion des clôtures dans cet interprète. Le but original
  était d'explorer la possibilité de stocker tout l'environnement courant (donc
  en le dédoublant), possibilité non développée en TD. Après constatation des
  nombreux problèmes que cela pose (mais aussi de ce qu'il est possible de
  faire) nous avons créé une autre clôture utilisant des pointeurs
  d'environnement.
\item Implémenter la gestion de frames, adapter la clôture \texttt{Scheme}.
\end{itemize}


\section{Travail réalisé}

\subsection{Interprète dynamique}

\subsubsection{Toplevel et environnement}

  Nous avons encapsulé la gestion des entrées dans un \texttt{toplevel}, 
 appelé par la fonction \texttt{main}. Ce toplevel gère la directive \texttt{setq}, qui
 permet de modifier l'environnement. 
 Elle fait appel à une fonction \texttt{add\_new\_binding} que nous avonsx
 implémenté, qui permet de créer un nouvel objet
  \texttt{Binding} (qui comporte deux éléments : un
 symbole et la valeur qui lui est associée) que l'on ajoute à l'environnement courant. 
 Dans le code original, les environnement sont traités comme des vecteurs de pointeurs 
 vers des
 \texttt{Binding}.

 \subsubsection{Subroutines}
 
  Dans le code fourni, quelques subroutines étaient traitées dans la fonction 
  \texttt{eval} de l'interprète. Nous avons encapsulé les subroutines pour qu'elles soient 
  traitées de manière séparée, afin de pouvoir facilement en ajouter de nouvelles. 
  Les opérations de bases sur les nombres, ainsi que les fonctions \texttt{car}, 
  \texttt{cdr}, \texttt{cons}, \texttt{concat}, \texttt{newline} et \texttt{read} 
  ont été implémentées.

  Il a par contre été plus difficile de passer des fontions en objet. En Ocaml,
  \og functions are first-class citizens \fg. Pas en C++. De fait, il a été
  nécéssaire de prendre des moyens détournés.
  Pour nous rapprocher de l'interprète original, nous avons modifié la classe
  \texttt{Cell} pour créer un type de cellule contenant une subroutine (qui est
  une fonction). Pour ce faire, nous avons fait usage des pointeurs de fonctions
  (et non de foncteurs dont l'utilisation ne se justifiait pas vraiment ici).
  Une fonction \texttt{init\_subr} permet d'ajouter à l'environnement de départ
  les subroutines définies. Il convient de noter qu'on a du modifier les
  méthodes d'impression de liaisons pour ne pas imprimer les liaisons faisant
  intervenir une subroutine.
  
 \subsubsection{Eval}
 
 La principale modification apportée à la classe \texttt{eval} fournie est 
l'utilisation de l'encapsulation précédemment mentionnée pour mieux compartimenter 
les différentes parties de l'évaluation. Nous avons utilisé la programmation 
par exception afin de s'assurer que chacune des ces parties se déroulent 
correctement. Par exemple, lors de l'évaluation d'un symbole, on fait appel 
à une fonction qui parcours l'environnement courant afin de trouver une liaison 
correspondante, et renvoie une exception en cas d'échec. Cela nous permet de 
repérer lorsque l'utilisateur cherche à évaluer un symbole qui n'est pas attribué.

Nous avons aussi aussi implémenté un dispositif de trace via la commande 
\texttt{debug} et d'impression de l'environnement via \texttt{printenv}, 
pour permettre à l'utilisateur de suivre l'évolution de l'environnement courant.

\subsection{Environnement par listes chaînées}

Si l’implémentation des environnement comme vecteurs 
de pointeurs vers des \texttt{Binding} a le mérite de marcher, elle n'en est
pas moins problématique au sens où, pour copier un environnement, il est
nécessaire de copier l'intégralité de la liste des pointeurs, ce qui présente un
coût linéaire en la taille de l'environnement considéré!
La première idée pour remédier à cela est de faire de l'environnement un unique
pointeur vers le vecteur de pointeurs. Cela ne fait cependant que déplacer le
problème dans le cas où il faut copier aussi ce vecteur.
Aussi, il semble judicieux de transformer l’environnement en liste chaînée:
chaque bloc de la liste contient un unique \texttt{Binding} et un pointeur vers
le prochain bloc (éventuellement \texttt{NULL}). Il faut noter que nous ne perdons rien
en efficacité, puisqu'il fallait de toute façon parcourir tout le vecteur de
pointeurs pour trouver le \texttt{Binding} correspondant à une chaîne donnée.

\subsubsection{Blocs d’environnement}

Nous avons ainsi défini une structure \texttt{EnvBlock} qui constitue un maillon
de la liste chaînée. Puisque ces maillons sont eux mêmes repérés par leur
pointeur, et pour une raison que nous verrons plus tard, il n'est pas nécessaire
d'utiliser des pointeurs de \texttt{Binding}.

\subsubsection{Environnement = Bloc?}

La principale question qui s'est posée est la suivante: considère-t-on qu'un
environnement est un bloc (le bloc de \og tête \fg) ou considère-t-on que
l'environnement est quelque chose d'autre que la somme de ses
parties individuelles?

Nous avons privilégié la deuxième approche. En effet, elle permet de gérer plus
proprement l'allocation dynamique en mémoire de blocs d'environnement. De plus,
cela rendait beaucoup plus facile la destruction des blocs d'environnements
alloués en mémoire ainsi que la création de copies indépendantes (créant aussi
de nouveaux blocs). Cette gestion des copies s'est révélée par la suite sans
grand intérêt et avait principalement un sens dans l'optique d'un dédoublement
de l'environnement au moment de la clôture (approche depuis abandonnée), mais le
code utilisé demeure dans le dossier \texttt{Copie}

\begin{attention}
  Une difficulté rencontrée avec l'utilisation d'instances de \texttt{Binding}
  au lieu de pointeurs dans les blocs d'environnements a été la création
  implicite de copies que nous avons du tracer à l'aide d'impressions de
  pointeurs. L'utilisation d'un setter a permis de remédier à ces problèmes.
\end{attention}

\subsection{Première liaison statique}

Nous nous sommes dans un premier temps contentés de créer une nouvelle méthode
\texttt{set\_new\_binding} qui parcours l'environnement à la
recherche d'une liaison faisant intervenir le nom de liaison qu'on veut ajouter.
Si on trouve une telle liaison, on modifie physiquement la liaison trouvée.
Sinon on rajoute une nouvelle liaison.

\subsection{Première clôture}

\subsubsection{Stockage physique de la capture}

Nous étions dans un
premier temps partis dans l'idée d'essayer de voir ce que permettait le stockage complet de
l'environnement courant au moment de la définition d'une clôture. La création
d'une telle clôture, bien que fastidieuse ne pose pas de difficultés
particulières: la principale difficulté rencontrée a été le fait qu'on ne peut
pas passer de \texttt{Binding} en argument dans un type union. Il a donc été
nécessaire d'improviser à partir du type pair pour stockers des liaisons. Avec
cette façon de procéder, les fonctions sont typiquement protégées de leur
environnement d'exécution.

Là où la situation devient plus épineuse est quand on veut pouvoir modifier
physiquement une variable déjà définie. Si il peut paraître en première approche
qu'il \og suffirait\fg de parcourir toutes les clôtures pour modifier les
informations à l'intérieur, cette idée n'est pas concluante  puisqu'une telle clôture
rendrait l'environnement circulaire. Nous n'avons pas trouvé de solution
satisfaisante à ce problème sans simplement passer aux pointeurs.

En l'état, une telle construction ne permet pas l'utilisation de fonctions
mutuellement dépendantes (on ne peut pas aller modifier le placeholder) ou
récursives.

\subsubsection{Stockage d'un pointeur de capture}

Nous avons donc du reprogrammer la clôture en stockant un pointeur vers le bloc
environnement courant au moment de la clôture. Il suffira alors de créer un
environnement pointant vers le bloc d'environnement au moment de la reconstruction.
Une légère difficulté est que simplement ajouter une valeur pointant vers un
environnement peut créer une situation où il y a des inclusions circulaires de
fichiers puisque la classe \texttt{Environment} dépend de la classe \texttt{Cell}.
Heureusement, une forward declaration de la classe \texttt{Environment} dans la classe
\texttt{Cell} permet de remédier à cela.

Avec cette méthode, il est possible de créer des fonctions mutuellement
récursives et des fonctions récursives sans problème apparent. Par contre il
faut encore gérer un problème posé par la définition circulaire des clôtures.
En effet, cela pose de gros problèmes si on veut afficher l'environnement. La
solution adoptée a été le remplacement d'objets environnement par une chaîne
\texttt{<env>}.

\subsection{Passage en frames}

Notre structure par listes chaînées a déjà grandement abordé les difficultés de la
gestion d'un environnement par frames. Deux possibilités d'adaptation étaient
envisageables: on pouvait laisser la structure grandement inchangée et
simplement remplacer le contenu des blocs d'environnements par un vecteur de
pointeurs vers des \texttt{Binding}, ou traiter les anciens
\texttt{Environment} comme des frames et créer un nouvel environnement pointant
vers une frame. C'est la deuxième possibilité que nous avons privilégié.
Procéder de la sorte nous semble présenter certains avantages: de cette manière
il est complètement naturel de créer des frames intermédiaires, et il y a
séparation claire des structures de stockage de liaisons et de frames.

Un aspect particulièrement intéressant de la gestion par frame est la limitation
de l'action de certaines méthodes (comme \texttt{set\_new\_binding}) à la frame
locale tandis que d'autres s'appliquent à l'ensemble des frames en cascade
défini par l'enchaînement des frames.

\subsection{Clôtures Scheme}

La gestion des clôtures en Scheme est comparativement plus simple que dans
l'interprète à liaison statique précédent. Il suffit de stocker le pointeur
vers la frame courante au moment de la clôture et il ne sera plus possible de
capturer une variable au cours de l'exécution de fonctions imbriquées.

Lors du passage par un \texttt{lambda}, une nouvelle frame est créée (ce qui
protège les variables de l'environnement courant), et à la lecture d'une clôture
on se replace dans la frame où la fonction a été définie ce qui évite d'être
gêné par une définition interne à la frame où la fonction est appelée.

Grâce à la particularité des frames, il est parfaitement naturel de définir des
fonctions mutuellement récursives. En effet, puisque la clôture pointe vers une
frame et non un bloc d'environnement fixé, il n'y a pas besoin d'utiliser un
placeholder pour réserver une place dans la clôture : les fonctions définies
dans la frame courante après la définition de la clôture seront quand même
capturées.

\section{Vérification}
 
 \subsection{Interprète dynamique}

 Les fonctions de l'interprète dynamique présenté en exemple dans le premier TD
 de Lisp fonctionnent correctement et permettent 
 une utilisation classique de Lisp.  
 Le seul problème est la gestion des erreurs de 
 l'utilisateur via les exceptions. La programmation défensive utilisée dans le 
 code original permet de s'assurer que tout fonctionne comme prévu, mais en cas 
 d'erreur, la fonction \texttt{assert} qui a été utilisée stoppe le processus de 
 l'interprète et empêche de rattraper certaines exceptions comme nous voulions le faire.
 Cet interprète permet aussi les dépassement d'entier, ce qui est étonnant car 
 les calculs sur les nombres, effectués via le code C++, devraient lever une 
 exception lors de tels dépassements. Nous n'avons pas découvert la source de 
 ce problème.
 
\subsection{Gestion de la mémoire}

La gestion mémoire a été abandonné dans le code \texttt{Scheme} puisque cette 
problématique n'avait pas grand intérêt dans l'approche \texttt{Scheme}. Néanmoins,
nous avons tout de même effectué des tests dans le code \texttt{Copie} avec
l'exécution du
code lisp suivant:

\begin{lstlisting}
(setq f (lambda (a) (+ a 1)))
(f 3)
\end{lstlisting}

Valgrind montre une différence d'une centaine de bytes entre la version avec
gestion mémoire de l'environnement (surcharge du destructeur et de l'opérateur
de copie) et celle sans.

 \subsection{Interprète Scheme}

 Le travail effectué sur la version Scheme de l'interprète consistait à implémenter 
 une gestion plus efficace et naturelle des clotûre au travers des frames. Cet 
 objectif a été rempli et les frames semblent fonctionner comme prévu. Une nouvelle
 frame est créée à chaque clotûre, et la recherche de liaison au sein d'une clotûre 
 s'effectue dans la frame capturée puis dans les frames vers lesquelles elle pointe. 
 Cela permet la définition de fonctions récursive et mutuellement récursives sans 
 passer par des placeholders. Pour mieux nous en rendre compte, nous avons
 modifié l'impression d'environnement pour faire apparaître un chevron >> au
 passage à une nouvelle frame ce qui permet de bien vérifier la bonne création
 et manipulation des frames.
 
 De plus, la gestion par pointeurs permet de limiter la taille occupée par les frames 
 à une taille raisonnable. L'utilisation des pointeurs a cependant été la source de 
 nombreuses erreurs, notamment au niveau de la mémoire, ce qui a demandé un gros 
 travail de debug pour que cette gestion s'effectue comme nous l'espérions. Un travail 
 supplémentaire a aussi été fourni sur la programmation défensive et la gestion des
 assertionS, ce qui nous a permis de rattraper un certain nombre d'exceptions pour les 
 gérer directement à travers l'interprète.

 Nous avons vérifié rapidement la bonne marche des fonctions suivantes entre
 autre choses:

\begin{lstlisting}
(setq f (lambda () a)) | (f) retourne la valeur de a dans l'environement global

(setq g (lambda (a) (f))) | (g 3) retourne la valeur de a dans l'environement
global

(setq f (lambda (n) (if (= n 0) 0 (g (- n 1))))) | (f n) retourne 0 si n impair
(setq g (lambda (n) (if (= n 0) 1 (f (- n 1))))) | (g n) retourne 1 si n pair

(setq fact (lambda (n) (if (= n 0) 1 (* n (f (- n 1)))))) | retourne !n
\end{lstlisting}
 
 Cependant, cet interprète ne permet pas de gérer la continuation. Il faudrait pour 
 cela implémenter la construction call/cc de Scheme, ce qui demanderait un travail supplémentaire 
 conséquent.


\section{Contribution}

Ce projet nous a permis d'explorer les différentes possibilités d'implémentation 
Lisp, en particulier en ce qui concerne la gestion de l'environnement. Nous avons pu 
tester les différents niveaux de complexité liés à cette gestion, des liaisons physiques
de l'interprète dynamique aux frames Scheme qui permettent une gestion plus naturelle des
des clôtures.

La gestion de l'environnement, n'utilisant pas les structures pré-programmées de
la \texttt{STL} a été l'occasion de faire un travail plus bas niveau que ce qui
est normalement fait en C++, la preuve en étant la faible utilisation de
notion propres à C++ (par rapport à C). En ce sens, le code original écrit
pourrait très bien être adapté en une librairie C++ \texttt{Environment}

L'interprète ayant été codé en C++, une réflexion de notre travail fut la gestion de la 
mémoire et l’utilisation des pointeurs. Comme nous l'avons vu plus haut, utiliser 
des pointeurs peut générer de nombreuses erreurs et par conséquent alourdir grandement la charge 
de travail. C'est la raison pour laquelle nous avons expérimenté une clôture par stockage 
complet de l'environnement, qui bien que gourmande en allocation mémoire, offrait l'avantage 
d'éviter d'être confronté à ces problèmes. Cela nous a permis de constater qu'une telle 
implémentation avait elle aussi des limites, et était la source de problèmes plus théoriques 
(comme la modification d'une liaison dans une environnement circulaire). C'est cette réflexion 
qui nous a conduit à l'implémentation actuelle des frames, qui s'est montré plus efficace 
malgré les dangers liés à l'utilisation des pointeurs.

Mais l'utilisation des frames seules présente elle aussi des limites, puisqu'en l'état 
notre interprète ne peux pas gérer la continuation. Cela pourrait faire l'objet d'une 
prolongation de notre travail, qui permettrait à l'interprète de se hisser au niveau 
d'un interprète Scheme classique. Il serait aussi possible d'implémenter une évaluation 
paresseuse, qui permettrait notamment une optimisation des calculs sur les listes.

\section*{Conclusion}

L'implémentation des clôtures et liaisons par frames dans un interprète lisp en
C++ a posé des problèmes de définitions qui ont demandé de revoir légèrement le
squelette de code fournit, tout comme la manipulation de fonctions en tant
qu'objets qui a du, étrangement, se faire par la manipulation des structures
plus archaïques C (langage impératif). La notion de liaison statique et de clôture a posé de
grandes questions quant à son implémentation, en particulier sur la façon
d'implémenter une clôture en se passant de pointeurs, ce qui compliquerait
inutilement la clôture.
En retraçant les fonctions de test proposées dans les TD Lisp, nous avons tenté
d'établir le bon fonctionnement de notre interprète Lisp. De nombreux bugs ont
été mis en évidence de cette façon, et la mise en œuvre de mécanismes de trace
a été nécessaire pour retracer la source des problèmes. Cela a été l'occasion
de mieux comprendre le fonctionnement de la manipulation d'objets en C++, et la
façon dont un environnement fonctionnant par liaisons statiques se modifie.

Pour correspondre à un interprète \texttt{Scheme}, il serait nécessaire
d'implémenter la fonction \texttt{call/cc}. De plus, la gestion de la mémoire,
bien qu'abordée dans le cadre de la gestion des environnements demeure problématique 
puisque la cause du problème demeure plus profonde que les solutions
envisagées qui n'aident vraiment que dans le cas d'une clôture par copie complète.

\section*{Annexe: auto-évaluation}

\subsubsection*{Points forts}
\begin{itemize}
\item Gestion naturelle des récursions grâce aux frames
\item Émulation d'une manipulation des subroutines en tant qu'objets
\item Clôture par pointeur plus efficace qu'une clôture par copie
\end{itemize}

\subsubsection*{Points faibles}
\begin{itemize}
\item Gestion des fuites mémoires non satisfaisante
\item Manque de directives \texttt{call/cc} pour gérer la continuation
\end{itemize}

\subsubsection*{Améliorations possibles}
\begin{itemize}
\item Conception d'un garbage collector
\item Création d'une directive \texttt{call/cc}
\item Remplacement des asserts par des levées d'exceptions spécialisées.
\end{itemize}

\subsubsection*{Difficultés à anticiper}
\begin{itemize}
\item On travaille principalement sur une seule grande frame.
\item La structure par frames s'extrait de l'environnement d'évaluation par sa construction
\end{itemize}

\end{document}
